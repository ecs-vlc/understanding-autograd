\documentclass{beamer}
\usepackage{lmodern}
\usepackage{listings}
\usepackage{amsmath}
\usepackage{bm}
\usepackage{textpos} % package for the positioning

\usetheme{Copenhagen}
\hypersetup{pdfstartview={Fit}}

\title{Automatic Differentiation}
\author{Jonathon Hare}
\institute[]
{
  Vision, Learning and Control\\
  University of Southampton 
}
\date{}
\subject{Computer Science}
\useoutertheme{infolines}
\setbeamertemplate{headline}{} %remove headline
\setbeamertemplate{navigation symbols}{} %remove navigation symbols

\begin{document}
  \frame{
  \titlepage

\tiny{Much of this material is based on this blog post:
\url{https://rufflewind.com/2016-12-30/reverse-mode-automatic-differentiation}}
  }

\begin{frame}
\frametitle{What is Automatic Differentiation (AD)?}
To solve optimisation problems using gradient methods we need to compute the gradients (derivatives) of the objective with respect to the parameters.
\begin{itemize}
	\item In neural nets we're talking about the gradients of the loss function, $\mathcal{L}$ with respect to the parameters $\bm{\theta}$: 
	$\nabla_{\bm{\theta}} \mathcal{L} = \frac{\partial \mathcal{L}}{\partial \bm{\theta}}$
\end{itemize}
\end{frame}

\begin{frame}
	\frametitle{What is Automatic Differentiation (AD)?}
	\framesubtitle{Computing Derivatives}
There are three ways to compute derivatives:

\begin{columns}
	\column{.45\textwidth}
    \setbeamercovered{transparent}
    \begin{itemize}
    	\item<1,2> Symbolically differentiate the function with respect to its parameters
    	\begin{itemize}
    		\item by hand
    		\item using a CAS
    	\end{itemize}
    	\item<1,3> Make estimates using finite differences
    	\item<1,4> Use Automatic Differentiation
    \end{itemize}

    \column{.45\textwidth}
      \begin{block}{Problems}<2>
        Static - can't "differentiate algorithms"
      \end{block}
      \begin{block}{Problems}<3>
        Numerical errors - will compound in deep nets
      \end{block}
  \end{columns}
\end{frame}

\begin{frame}
	\frametitle{What is Automatic Differentiation (AD)?}

	Automatic Differentiation is:
	\begin{itemize}
		\item a method to get exact derivatives efficiently, by storing information as you go forward that you can reuse as you go backwards.
		\begin{itemize}
			\item Takes code that computes a function and uses that to compute the derivative of that function.
		\item The goal isn't to obtain closed-form solutions, but to be able to write a program that efficiently computes the derivatives.
		\end{itemize}
	\end{itemize}
\end{frame}

\begin{frame}[fragile]
\frametitle{Lets think about differentiation and programming}
\begin{columns}
	\column{.45\textwidth}
    \begin{example}<1->[Math] 
    \vspace{-1.5em}
    \begin{align*}
			x & = ? \\
			y & = ? \\
			a & = x \, y \\
			b & = \sin(x) \\
			z & = a + b
		\end{align*}
    \end{example}

	\column{.45\textwidth}
	\begin{example}<2->[Code]
			\begin{lstlisting}
x = ?
y = ?
a = x * y
b = sin(x)
z = a + b
			\end{lstlisting}
	\end{example}
	\end{columns}
\end{frame}

\begin{frame}
\frametitle{The Chain Rule of Differentiation}

\uncover<1->{
Recall the chain rule for a variable/function $z$ that depends on $y$ which depends on $x$:
\begin{equation*}
\frac{dz}{dx} = \frac{dz}{dy}\frac{dy}{dx}
\end{equation*}
}

\uncover<2->{
In general, the chain rule can be expressed as:
\begin{equation*}
	\frac{\partial w}{\partial t} = \sum_i^N \frac{\partial w}{\partial u_i} \frac{\partial u_i}{\partial t} = \frac{\partial w}{\partial u_1}\frac{\partial u_1}{\partial t} + \frac{\partial w}{\partial u_2}\frac{\partial u_2}{\partial t} + \dots + \frac{\partial w}{\partial u_N}\frac{\partial u_N}{\partial t}
\end{equation*}
where $w$ is some output variable, and $u_i$ denotes each input variable $w$ depends on.
}
\end{frame}

\begin{frame}
\frametitle{Applying the Chain Rule}
Let's differentiate our previous expression with respect to some yet to be given variable $t$:
\begin{align*}
	\frac{\partial x}{\partial t} & = ? \\
	\frac{\partial y}{\partial t} & = ? \\
	\frac{\partial a}{\partial t} & = x \frac{\partial y}{\partial t} + y \frac{\partial x}{\partial t}\\
	\frac{\partial b}{\partial t} & = \cos(x) \frac{\partial x}{\partial t}\\
	\frac{\partial z}{\partial t} & = \frac{\partial a}{\partial t} + \frac{\partial b}{\partial t}
\end{align*}


\uncover<2->{If we substitute $t=x$ in the above we'll have an algorithm for computing $\partial x/\partial x$. To get $\partial z/\partial y$ we'd just substitute $t=y$.
}
\end{frame}

\begin{frame}[fragile]
\frametitle{Translating to code I}
We could translate the previous expressions back into a program involving \emph{differential variables} \lstinline!{dx, dy, ...}! which represent $\partial x/\partial t, \partial y/\partial t, \dots$ respectively:

\begin{lstlisting}
dx = ?
dy = ?
da = y * dx + x * dy
db = cos(x) * dx
dz = da + db	
\end{lstlisting}

\uncover<2->{What happens to this program if we substitute $t=x$ into the math expression?}
\end{frame}

\begin{frame}[fragile]
\frametitle{Translating to code II}

\begin{columns}
\column{.45\textwidth}
\begin{lstlisting}
dx = 1
dy = 0
da = y * dx + x * dy
db = cos(x) * dx
dz = da + db	
\end{lstlisting}

\column{.45\textwidth}

\uncover<2->{The effect is remarkably simple: to compute $\partial z/\partial x$ we just seed the algorithm with \lstinline!dx=1! and \lstinline!dy=0!.}
\end{columns}
\end{frame}


\begin{frame}[fragile]
\frametitle{Translating to code III}

\begin{columns}
\column{.45\textwidth}
\begin{lstlisting}
dx = 0
dy = 1
da = y * dx + x * dy
db = cos(x) * dx
dz = da + db	
\end{lstlisting}

\column{.45\textwidth}

To compute $\partial z/\partial y$ we just seed the algorithm with \lstinline!dx=0! and \lstinline!dy=1!.
\end{columns}
\end{frame}

\begin{frame}[fragile]
\frametitle{Making Rules}

\begin{itemize}
	\item<+-> We've successfully computed the gradients for a specific function, but the process was far from automatic. 
	\item<+-> We need to formalise a set of rules for translating a program that evaluates an expression into a program that evaluates its derivatives.
	\item<+-> We have actually already discovered 3 of these rules:
\begin{lstlisting}
c = a + b   =>  dc = da + db
c = a * b   =>  dc = b * da + a * db
c = sin(a)  =>  dc = cos(a) * da
\end{lstlisting}
\end{itemize}

\end{frame}

\begin{frame}[fragile]
\frametitle{More rules}

These initial rules:
\begin{lstlisting}
c=a+b     =>  dc=da+db
c=a*b     =>  dc=b*da+a*db
c=sin(a)  =>  dc=cos(a)*da
\end{lstlisting}
can easily be extended further using multivariable calculus:
\begin{lstlisting}
c=a-b     =>  dc=da-db
c=a/b     =>  dc=da/b-a*db/b**2
c=a**b    =>  dc=b*a**(b-1)*da+log(a)*a**b*db
c=cos(a)  =>  dc=-sin(a)*da
c=tan(a)  =>  dc=da/cos(a)**2
\end{lstlisting}
\end{frame}

\begin{frame}
\frametitle{Forward Mode AD}
\begin{itemize}
	\item<+-> To translate using the rules we simply replace each primitive operation in the original program by its differential analogue.
	\item<+-> The order of computation remains unchanged: if a statement $K$ is evaluated before another statement $L$, then the differential analogue of $K$ is evaluated before the analogue statement of $L$.
	\item<+-> This is \textbf{Forward-mode Automatic Differentiation}.
\end{itemize}
\end{frame}

\begin{frame}[fragile]
\frametitle{Interleaving differential computation}

A careful analysis of our original program and its differential analogue shows that its possible to interleave the differential calculations with the original ones:
\begin{columns}
	\column{.45\textwidth}
\small{
\begin{lstlisting}
x  = ?
dx = ?

y  = ?
dy = ?

a  = x * y
da = y * dx + x * dy

b  = sin(x)
db = cos(x) * dx

z  = a + b
dz = da + db
\end{lstlisting}
}

\column{.45\textwidth}
\begin{block}{Dual Numbers}<2->
	\begin{itemize}
		\item<2-> This implies that we can keep track of the value and gradient at the same time.
		\item<3-> We can use a mathematical concept called a "Dual Number" to create a very simple direct implementation of AD.
	\end{itemize}
\end{block}

\end{columns}
\end{frame}

\begin{frame}
\frametitle{Forward Mode AD}

Let's try this for ourselves... Open the 

\end{frame}

\begin{frame}
\frametitle{Backward Mode AD}
\framesubtitle{A bit more information about this}
%More content goes here
\end{frame}
% etc
\end{document}